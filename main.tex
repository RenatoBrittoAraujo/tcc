\documentclass[12pt,openright,oneside,a4paper,english,french,spanish]{abntex2}

\usepackage{cmap}	
\usepackage{lmodern}	
\usepackage[T1]{fontenc}	
\usepackage[utf8]{inputenc}		
\usepackage{lastpage}		
\usepackage{indentfirst}
\usepackage{color}	
\usepackage{graphicx}	
\usepackage{units}
\usepackage[brazilian,hyperpageref]{backref}
\usepackage[alf]{abntex2cite}
\usepackage{bold-extra}
\usepackage{eso-pic}
\usepackage{pdflscape}
\usepackage{tabu}
\usepackage{graphicx}
\usepackage[table,xcdraw]{xcolor}
\usepackage{multirow}
\usepackage{float}
\usepackage{xspace}
\usepackage[ruled,linesnumbered]{algorithm2e}
\usepackage{pdfpages}
\usepackage{caption}
\usepackage{graphicx}
% \usepackage{tabularx}
% \usepackage{booktabs}

\usepackage{booktabs,makecell,tabularx}
\usepackage{siunitx}
\usepackage{adjustbox}
\usepackage{array,booktabs}
\newcolumntype{C}[1]{>{\centering\arraybackslash}p{#1}}
\renewcommand\theadfont{\small}
\newcolumntype{L}{>{\raggedright\arraybackslash}X}

\newcommand{\itractool}{DTEST\xspace}

%Comandos para revisão
\newcommand{\rev}[1]{{\color{green}#1}\xspace}
\newcommand{\del}[1]{{\color{red}#1}\xspace}

%Comandos para impedir linhas órfãs e viúvas
\clubpenalty=10000
\widowpenalty=10000

\renewcommand{\backrefpagesname}{Citado na(s) página(s):~}
\renewcommand{\backref}{}
\renewcommand*{\backrefalt}[4]{
	\ifcase #1 %
		Nenhuma citação no texto.%
	\or
		Citado na página #2.%
	\else
		Citado #1 vezes nas páginas #2.%
	\fi}%
% ---

\usepackage{config/customizacoes}

% Dados pessoais
\autor{TODO: Nome do aluno}
\curso{TODO: Nome do curso}

% Dados do trabalho
\titulo{TODO: Nome do TCC}
\data{TODO: Ano do TCC}
\palavraChaveUm{TODO: palavra chave}
\palavraChaveDois{TODO: palavra chave}

% Dados da orientacao
\orientador{TODO: Dr. Prof.  Nome do Professor}
\coorientador{TODO: Dr. Prof. Nome do Professor}

% Dados para a ficha catalográfica
\cdu{}

% Dados da aprovação do trabalho
\dataDaAprovacao{}
\membroConvidadoUm{}
\membroConvidadoDois{}

\definecolor{blue}{RGB}{41,5,195}
\makeatletter
\hypersetup{
     	%pagebackref=true,
		pdftitle={\@title}, 
		pdfauthor={\@author},
    	pdfsubject={\imprimirpreambulo},
	    pdfcreator={LaTeX with abnTeX2},
		pdfkeywords={abnt}{latex}{abntex}{abntex2}{trabalho acadêmico}, 
		colorlinks=true,       		% false: boxed links; true: colored https://www.overleaf.com/project/62d53c90f74f126aee5fa883links
    	linkcolor=blue,          	% color of internal links
    	citecolor=blue,        		% color of links to bibliography
    	filecolor=magenta,      		% color of file links
		urlcolor=blue,
		bookmarksdepth=4
}
\makeatother
\setlength{\parindent}{1.3cm}
\setlength{\parskip}{0.2cm}  
\makeindex


\begin{document}
\renewcommand{\bibname}{References}
% \renewcommand{\s}{References}

\frenchspacing 
\imprimircapa
\imprimirfolhaderosto*

\begin{resumo}[Abstract]
 \begin{otherlanguage*}{english}


   \vspace{\onelineskip}

   \noindent
   \textbf{Key-words}: TODO: keywords
 \end{otherlanguage*}
\end{resumo}

\begin{siglas}

\item [UnB] University of Brasília

\end{siglas}
\pdfbookmark[0]{\contentsname}{toc}
\tableofcontents*
\cleardoublepage


\textual


% 
% 1. contextualiza o seu leitor publico alvo
% 2. mostra um overview da literatura relacionada ao negocio
% 3. qual o problema que existe? (e vc prentende resolver parcial/totalmente)
% 4. portanto, quais os objetivos dessa pesquisa
% 5. como vc vai organizar estes pontos?
\chapter{Introduction}        
\label{ch:introducao}

\section{Context}\label{contextualizacao}

TODO

\section{Literature Review}

TODO

\section{Problem}

TODO

\section{Objective}

TODO

\section{Paper Organization}

This paper is organized as follows:

\begin{itemize}

\item \textbf{Section \ref{ch:introducao} - Introduction:} TODO
\item \textbf{Section \ref{ch:introducao} - Related Work:} TODO
\item \textbf{Section \ref{ch:introducao} - Research Structure:} TODO
\item \textbf{Section \ref{ch:introducao} - Research Proposal:} TODO
\item \textbf{Section \ref{ch:introducao} - Results:} TODO
\item \textbf{Section \ref{ch:introducao} - Analysis:} TODO
\item \textbf{Section \ref{ch:introducao} - Conclusion and Future Work:} TODO

TODO 

\end{itemize}



% PUXA DADOS DOS OUTROS
% Aqui voce vai buscar coisas. Procure sistematicamente coisas que tratem do assunto
% envolvendo o problema que vc quer resolver. Catalogue aqui tudo de relevante que vc
% achou e explique sua relevancia
% Lembre-se de informar tambem o processo que te levou a isso, tudo o que vc leu.
% vc nao precisa falar sobre tudo que leu, e sim indicar as coisas relevantes que vc
% encontrou no processo
\chapter{Related Work}
\label{ch:referencial}

\section{Initial Considerations}

\section{Final Considerations}




% INDIQUE COMO VOCE VAI PUXAR SEUS PROPRIOS DADOS
% ou seja, qual o processo do que vc vai fazer
% quais sao as etapas e criterios desse processo
\chapter{Research Structure}
\label{ch:metodologia}
\section{Initial Considerations}

\section{Methodology plan}

\subsection{Planning} 

\subsection{Data Collection} 

\subsection{Data Analysis} 

\subsection{Results} 

\section{Final Considerations}

% APENAS PARA TCC 1
% basicamente descreve o que vc vai fazer no tcc2
% --> opicional dependendo do seu caso
    \chapter{Research Proposal}
\label{ch:proposta}

\section{Initial Considerations}

\section{Research Planning}

\section{Data Collection}

\subsection{Bibliographic Research}

\section{Cronogram}


% O ATO DE PUXAR TEUS PROPRIOS DADOS
% explica cada passo como tu ta fazendo
% dados devem ser reproduziveis
\chapter{Results}

% ANALISA TUDO QUE VC TEM ATE AGORA
% chegou a hora de falar com propriedade
% extraia constants a partir de tudo que vc fez e analise
% o que voce extraiu. Desenhe o traco da sua nova descoberta
\chapter{Analysis}

% CONCLUA
% indique aqui notas finais sobre a busca realizada
% trace um caminho para possiveis pesquisas no futuro que podem
% usufruir do que vc fez aqui.
\chapter{Conclusion and Future Work}

\bookmarksetup{startatroot} 

\postextual

\bibliography{bibliografia}


\printindex
\end{document}

