\documentclass[12pt,openright,oneside,a4paper,english,french,spanish]{abntex2}

\usepackage{cmap}	
\usepackage{lmodern}	
\usepackage[T1]{fontenc}	
\usepackage[utf8]{inputenc}		
\usepackage{lastpage}		
\usepackage{indentfirst}
\usepackage{color}	
\usepackage{graphicx}	
\usepackage{units}
\usepackage[brazilian,hyperpageref]{backref}
\usepackage[alf]{abntex2cite}
\usepackage{bold-extra}
\usepackage{eso-pic}
\usepackage{pdflscape}
\usepackage{tabu}
\usepackage{graphicx}
\usepackage[table,xcdraw]{xcolor}
\usepackage{multirow}
\usepackage{float}
\usepackage{xspace}
\usepackage[ruled,linesnumbered]{algorithm2e}
\usepackage{pdfpages}
\usepackage{caption}
\usepackage{graphicx}
% \usepackage{tabularx}
% \usepackage{booktabs}

\usepackage{booktabs,makecell,tabularx}
\usepackage{siunitx}
\usepackage{adjustbox}
\usepackage{array,booktabs}
\newcolumntype{C}[1]{>{\centering\arraybackslash}p{#1}}
\renewcommand\theadfont{\small}
\newcolumntype{L}{>{\raggedright\arraybackslash}X}

\newcommand{\itractool}{DTEST\xspace}

%Comandos para revisão
\newcommand{\rev}[1]{{\color{green}#1}\xspace}
\newcommand{\del}[1]{{\color{red}#1}\xspace}

%Comandos para impedir linhas órfãs e viúvas
\clubpenalty=10000
\widowpenalty=10000

\input{config/comandos}
\usepackage{config/customizacoes}

% Dados pessoais
\autor{Renato Britto Araujo}
\curso{Software Engineering}

% Dados do trabalho
\titulo{A Collection Of Patterns For Safe Smart Contracts}
\data{2022}
\palavraChaveUm{Smart Contracts}
\palavraChaveDois{Patterns}

% Dados da orientacao
\orientador{Dr. Prof.  Rejane Maria da Costa Figueiredo}
\coorientador{Dr. Prof. Rejane Maria da Costa Figueiredo}

% Dados para a ficha catalográfica
\cdu{}

% Dados da aprovação do trabalho
\dataDaAprovacao{}
\membroConvidadoUm{}
\membroConvidadoDois{}

% Dados pessoais
\autor{Renato Britto Araujo}
\curso{Software Engineering}

% Dados do trabalho
\titulo{A Collection Of Patterns For Safe Smart Contracts}
\data{2022}
\palavraChaveUm{Smart Contracts}
\palavraChaveDois{Patterns}

% Dados da orientacao
\orientador{Dr. Prof.  Rejane Maria da Costa Figueiredo}
\coorientador{Dr. Prof. Rejane Maria da Costa Figueiredo}

% Dados para a ficha catalográfica
\cdu{}

% Dados da aprovação do trabalho
\dataDaAprovacao{}
\membroConvidadoUm{}
\membroConvidadoDois{}

\input{config/setup}

\begin{document}
\renewcommand{\bibname}{References}

\frenchspacing 
\imprimircapa
\imprimirfolhaderosto*

\begin{resumo}[Abstract]
 \begin{otherlanguage*}{english}


   \vspace{\onelineskip}

   \noindent
   \textbf{Key-words}: TODO: keywords
 \end{otherlanguage*}
\end{resumo}

\begin{siglas}

\item [UnB] University of Brasília
\item [EVM] Ethereum Virtual Machine
\item [NVD] A US Gov database for security vulnerabilities
\item [CVE] Another US Gov database for security vulnerabilities


\end{siglas}
\input{config/indiceAutomatico}

\textual


% 
% 1. contextualiza o seu leitor publico alvo
% 2. mostra um overview da literatura relacionada ao negocio
% 3. qual o problema que existe? (e vc prentende resolver parcial/totalmente)
% 4. portanto, quais os objetivos dessa pesquisa
% 5. como vc vai organizar estes pontos?
\chapter{Introduction}        
\label{ch:introducao}

\section{Context}\label{contextualizacao}

Blockchain is a technology that enables trust-less communication without the need of a third party. That’s achieved via a decentralized peer-to-peer network (EVM), in which its nodes (miners) are running a complete implementation of the protocols set by the blockchain as well as all the data in it. The data is in the form of a ledger, and this ledger can contain entries for raw machine code - these are known as smart contracts, compiled from the Solidity programming language, and they operate similar to how a class would: its interface exposed to the outside world via method calls. When called, the functions are executed for a fee paid to the nodes that execute the computation separately. Smart contracts enable enforcement of agreed terms between two or more untrusted parties and once one is deployed, it cannot be revoked and becomes a permanent part of the ledger. Finally, a smart contract may hold real money just like a class may hold an integer or string. Because of all these properties, smart contracts’ code errors and vulnerabilities can be disastrous, and prevention against them a must.  

\section{Literature Review}

Several papers tackle security risks and protection for smart contracts. In \cite{chen2020survey}, \cite{kushwaha2022systematic} and \cite{sayeed2020smart}, an extensive study and analysis is performed over existing literature and vulnerability databases (such as NVD or CVE) to detect and categorize them with rigor. Furthermore, \cite{singh2020blockchain}, \cite{ali2021sescon} and \cite{vivar2021security} addresses approaches to automated and manual detection and frameworks for designing secure smart contracts.  
asdasd


\section{Problem}

Smart contract programming is a new field, with little to no patterns, which enable the developers to creatively (and dangerously) design their contracts. Some designs for a smart contract are safer than others, and contracts made with ease of testing and understanding are more likely to successfully avoid corruption or having funds robbed. Despite several audits, the more complex a project is, the more likely a vulnerability is to be found, so safety is a matter of caution for the smallest to biggest institutions creating smart contracts. Several automated vulnerability detection tools exist, but the architecture of a contract can fail regardless.

\section{Objective}

With the intent to provide smart contract programmers a set of canned, battle-tested smart contract design patterns, this study identifies, compiles and investigates multiple contracts using several analysis frameworks and well-known vulnerability lists to extract similarities in smart contracts which tend to yield higher security independent of their domain.


Asdasdasdsa
The work is composed of several parts which construct on top of the other:

\begin{enumerate}
\item \textbf{Vulnerability Analysis Accumulation:} firstly, this research accumulate a body of vulnerabilities and frameworks for detecting them. To construct this, a systematic literature review of vulnerabilities and vulnerability prevention methods is performed targeting the question: \textit{what makes a smart contract unsafe?} 
\item \textbf{Sample Gathering}: through online platforms in which open-source repositories can be found, such as Github or Gitlab, a list of well-known and frequently used smart contracts, through which a significant amount financial assets are managed, is compiled. The reason for such constraints is to ensure that by their exposure to a wide audience and real rewards for exploitation, their continued untroubled operation means they are safer and the large work that was put into them gets analysed for further usage.
\item \textbf{Pattern Identification}: via cross-referencing analysis over the smart contracts, shared patterns should naturally emerge. These patterns will be found and listed in relation to the problem it tries to solve on a abstract level.
\item \textbf{Pattern Testing}: A final list of patterns is tested against the analysis framework on step 1, to evaluate their security when solving a specific common need. These patterns also get evaluated on other practical dimensions such as scalability and price.
\end{enumerate}

A non-goal for this paper, therefore, is not to accumulate the latest research and discoveries related to smart contract vulnerabilities and prevention methods, but to identify patterns in smart-contracts that have not yet been successfully exploited after it's had ample chance of such an event happening.

\section{Paper Organization}

This paper is organized as follows:

\begin{itemize}

 \item \textbf{Section \ref{ch:introducao} - Introduction:} The context, research problem and objective, and a methodological synthesis;
 
    \item \textbf{Section \ref{ch:referencial} - Related Work:} Includes works that contribute to the conclusions of this paper.
    
    \item \textbf{Section \ref{ch:metodologia} - Methodology:} Studies the methodological route taken in depth.
       
       
    \item \textbf{Section \ref{ch:proposta} - Investigation:} The information gathering process, pattern identification and analytical testing.
       
       
    \item \textbf{Section \ref{ch:proposta} - Proposals:} Presents patterns for safe smart contract development.

\end{itemize}



% PUXA DADOS DOS OUTROS
% Aqui voce vai buscar coisas. Procure sistematicamente coisas que tratem do assunto
% envolvendo o problema que vc quer resolver. Catalogue aqui tudo de relevante que vc
% achou e explique sua relevancia
% Lembre-se de informar tambem o processo que te levou a isso, tudo o que vc leu.
% vc nao precisa falar sobre tudo que leu, e sim indicar as coisas relevantes que vc
% encontrou no processo
\chapter{Related Work}
\label{ch:referencial}

\section{Initial Considerations}
In this chapter, an analysis of the body of knowledge  

\section {Vulnerabilities}


\section {Vulnerability Prevention}


\section{Final Considerations}




% INDIQUE COMO VOCE VAI PUXAR SEUS PROPRIOS DADOS
% ou seja, qual o processo do que vc vai fazer
% quais sao as etapas e criterios desse processo
\chapter{Research Structure}
\label{ch:metodologia}
\section{Initial Considerations}

\section{Methodology plan}

\subsection{Planning} 

\subsection{Data Collection} 

\subsection{Data Analysis} 

\subsection{Results} 

\section{Final Considerations}

% TCC 1, foda-se
\input{src/proposta}

% O ATO DE PUXAR TEUS PROPRIOS DADOS
% explica cada passo como tu ta fazendo
% dados devem ser reproduziveis
\input{src/desenvolvimento}

% ANALISA TUDO QUE VC TEM ATE AGORA
% chegou a hora de falar com propriedade
% extraia informacoes a partir de tudo que vc fez e analise
% o que voce extraiu. Desenhe o traco da sua nova descoberta
\input{src/analise}

% CONCLUA
% indique aqui notas finais sobre a busca realizada
% trace um caminho para possiveis pesquisas no futuro que podem
% usufruir do que vc fez aqui.
\input{src/conclusao}

\bookmarksetup{startatroot} 

\postextual

\bibliography{bibliografia}


\printindex
\end{document}

